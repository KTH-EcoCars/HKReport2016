\chapter{Optimization}
\section{Introduction}
The main goal of the project is to have the car consume the least amount of fuel
possible.  In the ideal world shall the engines and motors therefore always
operate within a certain range of their optimal point of operation. The torque
request from each of the engine/motors should consider this in order to reduce
fuel consumption. In addition to this goal, there is a number of constraints
that needs to be considered in addition to the generic fuel consumption
optimization.  Things like track topography, super-capacitor voltage,
competitors on the track etc. will all weight in on the output request. The
optimization algorithm used in Elbas automatic drive mode was developed as a
part of a Master Thesis~\cite{liu2016}. The control architecture for this
optimization consists of two or three layers and the thesis have completed the
top layer speed reference control. This chapter will cover the decentralized
hierarchical predictive control that is used to calculate the speed trajectory
and it's implementation.

\section{Constraints}
The optimization is limited by the constraints of the physical system, meaning that
the controller is not able to operate outside the limits of the system. These limits
are,
\begin{align}
    &U_{cap} \in [39,48]~\textup{V} \\
    &\nu \in [0,15]~\textup{m/s}\\
    &a_{\max} \in [-1,1]~\textup{m/s$^2$}
\end{align}
where $U_{cap}$ is the voltage level of the super-capacitor, $\nu$ is the speed
and $a$ is the acceleration in every time step. Due to the fact that there is a
lower limit to the voltage level of the super-capacitor, it is necessary to
change drive mode to engage the ICE in order to regenerate and store energy
until the upper limit is reached. Since the optimization is currently only done
in the top layer, there is a need for the voltage limit to be considered
manually in the model.

\section{Position recognition}
The optimization bases the output reference speed on the current position on the
track. This yields the need for a position recognition with an accuracy precise
enough to be able filter it with sufficient results. The track itself, rather
then time is discretized in order to get a more accurate position in relation to
the tracks topography. \\
There is two ways of finding Elba's position on the track. The first way is a dead
reckoning system based on the shaft encoder~\cite[p.~49]{elba2015},
which calculates the position based on the rotational speed of the shaft. The
second way is based on the GPS system implemented in the race capture pro 2,
where the GPS signal is taken from it's raw form and reshaped with a
Lua-script\footnote{Lua is a dynamically typed scripting language, www.lua.org}
into a CAN-message that is sent on the CAN-bus.

\section{Simulink implementation}
The complete description of the algorithm is found in~\cite{liu2016} and the
implementation is based on this thesis.
\begin{figure}[H]
    \centering\label{fig:optimization_cont}
    \includegraphics[width=\textwidth]{./img/optimization_cont.png}
    \caption{Simulink implementation for the Hierarchal Optimization Algorithm}
\end{figure}
In Figure~\ref{fig:optimization_cont} the function
\textit{reset\_track\_distance} takes the absolute dead reckoning position
calculated from the shaft encoder and an indicator for a new
lap,~\textit{new\_lap} as input. When the indicator for a new lap is pressed,
the dead reckoning position is zeroed in the function block and the position is
calculated and delivered as the output. The position is sent to a Particle
Filter block where both the dead reckoning and the GPS position is inputed.
These two positioning systems should to be used to get the estimation of
the traveled distance the car have traveled on the track. The reason for using
th particle filter is because the encoder will likely have a drift and it is
a known that there is an inaccuracy in the GPS. However, the
Particle Filter itself is not implemented at the moment and is considered to be
something future groups need to work on.\\
There is three inputs into the three dimentional optimization lookup-table, the first one is a
time. This input is used to synchronize the different time scales in the table and
Simulink. The speed input is read from the CAN-bus and is the current speed of
Elba. The track position is the calculated distance traveled on the track from
the filter. The output of the block is an acceleration reference which needs to
be converted to an reference speed since Elba is currently controlled with speed
request. 

