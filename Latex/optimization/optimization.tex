\chapter{Optimization}
\section{Introduction}
The main goal of the project is to have the car consume the least amount of fuel
possible. In the ideal world should the engines and motors therefor always operate
within a certain the range of their optimal point of operation. The torque request
from each of the engine/motors should be with this goal in mind in order to reduce
fuel comsumption. In addition to this goal, there is a number of constraints that
needs to be considered on top of this generic optimization.  Things like track
topography, super-capacitor voltage, competitors on the track etc will all weigh in
on the output request. The optimization algorithm used ELBA was developed as a part
of a Master Thesis~\cite{lui2016}. The control architecture for ELBA consists of a
two or three layers and the thesis have completed the top layer speed reference
control. This chapter will cover the decentralized hierarchical predictive control
that is used to calculate the speed trajectory.

\section{Constraints}
The optimization is limited by the constraints of the physical system, meaning that
the controller is not able to operate outside the limits of the system. These limits
are,
\begin{align}
    &U_{cap} \in [39,48]~\textup{V} \\
    &\nu \in [0,15]~\textup{m/s}\\
    &a_{max} \in [-1,1]~\textup{m/s$^2$}
\end{align}
where $U_{cap}$ is the voltage level of the super-capacitor, $\nu$ is the speed and
$a$ is the acceleration in every time step. Due to the fact that there is a lower
limit to the voltage level of the super-capacitor, it is necessary to change
drive mode to the ICE in order to regenerate and store energy until the upper limit is
reached. 

\section{Algorithm}

\section{Simulink implementation}
