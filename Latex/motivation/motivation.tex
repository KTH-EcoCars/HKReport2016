\chapter{Purpose}

Why is is this project even interesting?

Lowering fuel consumption is one of the main challenges for the automotive 
industry. Regularly new standards for permitted emissions are passed and  
one solution is to burn less fuel, which results in less emissions.

Hybrid vehicles have been growing in popularity over the last decade and 
more lately have the plug in hybrids emerged. These cars allows one 
to charge the car with electrical energy at home, potentially at a low price.

Elba can be called a plug in hybrid and reducing fuel (and energy) consumption is the goal
of SEM\@. But since the style of hybrid is very unique, with the triple motor/ engine
combination, the problem of reducing the fuel consumption is not trivial. 

\section{The Problem}
When driving any car around a track some parameters and variables have a large effect of
how much energy is used to propel the car forwards. Speed, weight, drag and friction are
the first that comes into mind. Both drag and friction are increased with speed.

\section{Applications}

\section{Wall of text}
If one wishes to perform well in Shell Eco-Marathon optimising the speed the car should have on the track could improve performance. 
This is also something that is interesting for the industry and is not use in current cars when using autimatic speed thingys. This would both save money and the environment due to the reduced fuel consumption.

One goal the optimisation is to improve result at SEM. Therefore a car, in competition shape, has to be maintained. The car has to be updated to comply whit the current SEM-rules and also simply to be cemt in a drivable state. The car is also seen as a platform for students to test ideas and perform research. Various parts of the car have already been developed as bachelors thesis's and the optimisation will be improved as a part of a master thesis. The car enables students to try to solve both complex construction and design tasks, such as the unique clutch or the laminate door, and experiment with different optimisation strategies in a real world case. Appearing and performing well with the car is the main way to get sponsorship deals to support the project, both with parts and with money to cover travel and transportation costs.

But to be able to do all this test has to be carried out with as little overhead(time, track, transportation etc.) as possible. Also the actual competition track is located in London (and it only exists during the event), so there is no option to go there and test. This is the motivation behind the testrig. Other teams competing in SEM uses these machines to test the car while standing still, but KTH EcoCars had no machine like this. The testrig solves several practicalities for KTH EcoCArs and is therefore not only benificial for the car that was worked on, Several problems had to be solved for it to work: is must supply the torque demanded, is must simulate the forces the car is exposed to, and is has to handle supply of energy and regenerated energy created when braking the car.