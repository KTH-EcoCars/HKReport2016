\chapter{Elba The Car}
The hybrid vehicle that was build upon was named Elba several years ago when it was still a pure electric vehicle. The platform has been continuously improved and altered to the complex hybrid that exist today. Each year Elba participates in Shell Eco Marathon as an "Urban Concept Vehicle".

\section{Introduction (?)}
\subsection{How it moves}
Elba combines one DC motor, one Brushless DC motor (BLDC) and a one-cylinder internal combustion engine; to provide torque to the rear right wheel, propelling the car forwards.
A single person can ride the car, that person is coincidentally, also the driver. 
The driver steers the car and decide upon a reference speed that the car should follow and has the option to decide what mode (what combination of motor and engine) that should be used at a specific time.

\subsection{Aim}
The aim of Elba is to provide a platform to perform various levels of projects for engineering students, while at the same time be allowed to compete in Shell Eco Marathon.

\subsection{Urban Concept Vehicle}

\section{System Overview}
\subsection{Electronic Control Units}
The car has four electronic control units (ECUs) that are responsible for different parts of the car.

\begin{itemize}
\item Front-ECU
\item Back-ECU
\item ICE-ECU
\item Clutch-ECU
\end{itemize}

The Front-ECU is responsible for the human interface, the buttons where the driver is able to decide mode and speed if the car is in manual mode. It sends speed and dive mode on the can bus.

The Back-ECU controls the motors via the separate motor controllers. It has veto over the ICE and Clutch.

The internal combustion engine is controlled by the ICE-ECU and this is its only task. It senses the lambda value and motor? temperatures to control the injection of fuel.

The Clutch-ECU controls the clutch. It interfaces to two H-bridges which in turn steers the linear actuators.  

\subsection{Communication}
The car uses a CAN network to communicate between the distributed micro-controllers. The micro-controllers have specific values that are needed elsewhere; these are sent on the CAN-bus. 

Another CAN network is used between the Back-ECU and the inMotion driver that controls the brushless motor. This network ustes the protocol CANopen that lies on top of the CAN bus.
The instrumentation panel communicates via Bluetooth with the data-logger/GPS-unit.

\subsection{Energy supply}
All energy used to propel the car forwards (during the competition) comes from the ethanol fuel, but electrical energy can be stored in the super-capacitor. When the electrical motors generate energy it can be stored in this super-capacitor for later use. The car can only start from standstill using one of the electrical motors with energy from the super-capacitor.  

\section{Requirements}
\subsection{Rules}
At Shell Eco Marathon all vehicles must pass two inspections before the vehicle is allowed to enter the track. First a safety inspection evaluating if it's dangerous to have the vehicle on the track, both for the driver and other vehicles. And second a technical inspection asserting if the vehicle complies with the competition rules.

\section{Design decisions}
\subsection{Clutch decisions}

\subsection{Internal Combustion Engine}

\section{Results}
\subsection{Competition}

\subsection{End of project}

\section{Conclusion}