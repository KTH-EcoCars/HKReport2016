\chapter{Elba The Car}
The hybrid vehicle that was build upon was named Elba several years ago when it was still a pure electric vehicle. The platform has been continuously improved and altered to the complex hybrid that exist today. Each year Elba participates in Shell Eco Marathon as a "Urban Concept Vehicle".

\section{Introduction (?)}
\subsection{How it moves}
Elba combines one DC motor, one Brushless DC motor (BLDC) and a one-cylinder internal combustion engine; to provide torque to the rear right wheel, propelling the car forwards.
A single person can ride the car, that person is coincidentally, also the driver. 
The driver steers the car and decide upon a reference speed that the car should follow and has the option to decide what mode (what combination of motor and engine) that should be used at a specific time.

\subsection{Aim}
The aim of Elba is to provide a platform to perform various levels of projects for engineering students, while at the same time be allowed to compete in Shell Eco Marathon.

\section{System Overview}
\subsection{Communication Network}
The car uses a CAN network to communicate between the distributed micro-controllers.

\subsection{Energy supply}

\section{Requirements}
\subsection{Rules}
At Shell Eco Marathon all vehicles must pass two inspections before the vehicle is allowed to enter the track. First a safety inspection evaluating if it's dangerous to have the vehicle on the track, both for the driver and other vehicles. And second a technical inspection asserting if the vehicle complies with the competition rules.

\section{Design decisions}
\subsection{Clutch decisions}

\subsection{Internal Combustion Engine}

\section{Results}
\subsection{Competition}

\subsection{End of project}

\section{Conclusion}