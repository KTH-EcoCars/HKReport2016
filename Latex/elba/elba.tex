\chapter{Elba The Car}
The hybrid vehicle that was build upon was named Elba several years ago when it
was still a pure electric vehicle. The platform has been continuously improved
and altered to the complex hybrid that exist today. Each year Elba participates
in Shell Eco Marathon (SEM) as an ``Urban Concept Vehicle''.

%Introduction
\section{Introduction}
This section gives an introduction to Elba
\subsection{Propulsion}
Elba is an electric-ethanol hybrid vehicle which combines two 48V electrical
motors and one internal combustion engine (ICE). The electrical motors is one
0.2 kW direct current (DC) motor connected directly to the drive shaft and one
1.1 kW brushless DC motor (BLDC) connected via the clutch to the drive shaft.
The one-cylinder four-stroke ethanol powered ICE is connected to the drive
shaft via the clutch. These three provide a driving torque to the rear drive
shaft. The drive shaft is then connected to the right rear wheel via an 1:10
ratio planetary gear. The car fits a single human, he or she is also the
driver. The driver steers the car manually and decides upon a reference speed
that the car should follow and has the option to decides what motor/engine
combination that should be used at a specific time.

\subsection{Aim}
The aim of Elba is to provide a research platform to perform various levels of
projects for students from different science fields, while at the same time be
accepted to compete in SEM\@. The projects can vary in time and
extent and can be independent or made as an long term implementation. 

\subsection{Shell Eco-Marathon}
SEM is a vehicle competition focused on fuel efficiency where engineering
students from around the world designs, builds and drives their own vehicles.
The competition is split into two classes. The Prototype class is solely
focused on fuel-efficiency and have less restrictions concerning comfort and
usability compared to the second class which is the UrbanConcept class,
see~\ref{UCV}.

\subsection{Urban Concept Vehicle}\label{UCV}
ELBA competes in the UrbanConcept class and is what is called a UrbanConcept
Vehicle (UCV). The vehicles in this class have an appearance closer to today's
production type passenger cars. The cars in the class have to be built in
accordance with the class specific rules that dictates everything from steering
and controls to propulsion and safety. UCVs must have some common production
car features as wind shield wiper, turn signals, horn headlights etc. Vehicles
competing in this group is also required ``stop and go'' each lap, which means
that once every lap the vehicle needs to do a full stop. This is to further
increase the resemblance to city driving.

\subsection{State at start of project}
The car where in working condition when the project started, but had several
problems. To mention a few major ones: The clutch where slow and unreliable,
the overall power-output might not have been enough to be efficient on the new
track, none of the potential driver would fit in the driver compartment, the
horn disturbed the entire electrical system.

These issues where, together with the competition rules, the primary base for
the decisions that where taken and related to the car. Some of the work done to
``fix'' these problems where a matter of reshape some metal piece and bolting it
back in the car, suddenly the driver could reach the brakes. While other
required an entire team of people to get a new, working, ethanol engine in the
car.

%System Overview
\section{System Overview}
\subsection{Electronic Control Units}
The car has four electronic control units (ECUs) that are responsible for
controlling different parts of the car.

\begin{itemize}
\item Front-ECU
\item Back-ECU
\item ICE-ECU
\item Clutch-ECU
\end{itemize}

The Front-ECU is responsible for the human interface, the buttons where the
driver is able to decide mode and speed if the car is in manual mode. It sends
speed and dive mode on the can bus.

The Back-ECU controls the motors via the separate motor controllers. It has
veto over the ICE and Cluch.

The internal combustion engine is controlled by the ICE-ECU and this is its
only task. It senses the lambda value and motor? temperatures to control the
injection of fuel.

The Cluch-ECU controls the clutch. It interfaces to two H-bridges which in turn
steers the linear actuators.  

\subsection{Communication}
The car uses a CAN network to communicate between the distributed
micro-controllers. The micro-controllers have specific values that are needed
elsewhere; these are sent on the CAN-bus. 

Another CAN network is used between the Back-ECU and the inMotion driver that
controls the brushless motor. This newtwork ustes the protocol CANopen that
lies on top of the CAN bus.
The instrumentation panel communicates via Bluetooth with the data-logger/GPS-unit.

\subsection{Energy supply}
All energy used to propel the car forwards (during the competition) comes from
the ethanol fuel, but electrical energy can be stored in the super-capacitor.
When the electrical motors generate energy it can be stored in this
super-capacitor for later use. The car can only start from standstill using one
of the electrical motors with energy from the super-capacitor.  


\section{Software and Simulation Models}
Using a model based approach relies on verified systems models on various levels. Elba benefits form a full system model to simulate fuel efficency and each ECU has a corresponding simulink model from which all code is compiled. 
\subsection{Plant Model}
The plant model is supposed to be a full system model capable of estimating fuel consumption over an entire SEM attempt. When any component, which has an affect on propulsion, is changed on the car the plant model must also be updated. Naturally the goal with MBD is that any change can be simulated with the plant model, approved and then the car is updated. But not all simulations can be done beforehand, which was the case for the new ICE.\@

\subsection{ECU software}
All ECUs use an Arduino (Due or Mega) as micro-controller which simulink has
compiler support for. This means that simulink models can be compiled and
uploaded to the arduino directly from simulink. The Simulink to Arduino coupling also
gives the option for running ``external mode'' a Hardware in the loop type compile mode.
This makes it possible to look at values and signals, running on the ECU, in
real-time.
 
%Requirements
\section{Requirements}
\subsection{Rules}
At Shell Eco Marathon all vehicles must pass two inspections before the vehicle is allowed to enter the track. First a safety inspection evaluating if it's dangerous to have the vehicle on the track, both for the driver and other vehicles. And second a technical inspection asserting if the vehicle complies with the competition rules.

%Design Decisions
\section{Design decisions}
\subsection{Clutch decisions}
%1. What the old team told us
%2. What we told MD to do
%3. What MD did
%4. Problems still
%5. Finding the problem
%6 Fixing the arms

One of the major problems with the car has been the clutch. Figure X illustrates the mechanical components of the drivetrain of elba.

The Problem stated when starting the project was that the clutch actuators were oversized and thus giving more force than necessary for the clutch to engage properly. When the clutch was engaged with too much force, the clutch plate on the ICE side stuck very hard to the center plate and could not be disengaged using the actuator.

To solve the mechanical problems it was decided to appoint one team from the machine design department to work solely on the mechanical parts of the clutch. The mechatronics team decided to completely remake the clutch ECU to make it more robust. Despite the new ECU and the work performed by the machine design team, which can be read about in (TODO REPORT.XX) there was problems with the clutch when it was time to race. The problems was of the same mechanical nature as before the rework, where the clutch plate on the ICE side could not be disengaged from the drivetrain.

To solve this problem, several solutions have been investigated. The clutch mechanism has been controlled by time rather than position. One approach was to use the built in hall sensor in the actuator to control the position. One other idea was to use an external current sensor on the clutch ECU to control the position. A third idea was to continue using time as reference, but increase it so that an end position is always reached, which can be used as a reference position.

When investigating alternative three above using longer time to engage and disengage the clutch a new problem was discovered. 


\subsection{Internal Combustion Engine and ICE ECU}
It was decided that this years car should compete in the alternative fuels
class. This means that the car should be powered by an engine that runs on
ethanol. An engine running ethanol need to have a higher compression ratio than
a petrol engine.
%% TODO: Source on the above. Also, link to ICE report.
The requirements engineering and modifications on the new ICE was set by the ICE
team. To control the new engine, a new ICE ECU was needed. It was also decided
that the new ICE ECU would contain more sensor inputs as well as the possibility
to control the ICE ignition. All functionality of the old ICE ECU would still be
present, namely:
\begin{itemize}
    \item Simulink TLC software.
    \item Feedback control loop of injection time.
    \item Lambda value sensing.
    \item Encoder reading with index pulse from ICE\@.
\end{itemize}
More sensor data together with ignition control would give possibilities
for more exact control of the ICE\@. The extra features of the new ECU were:
\begin{itemize}
    \item Motor oil temperature sensing.
    \item Ambient air temperature sensing.
    \item Ignition control.
\end{itemize}
The features were implemented on the ECU PCB and the system was designed so that
the sensors and features could be implemented incrementally according to the
time and need of the project. Given that the amount of time that would be needed
to get the car in working condition before the race was not easily planned
beforehand, it was beneficial to have a base functionality with flexibility to
add features.

\subsection{Door}
The car is required to have a door, with a large enough entrance and working locking mechanism \cite{semrules16c1}. Since the old, layered carbonfibre door was too flimsy and also back hinged (there is a reason its call "suicide door"), it was decided a new, sturdier, door would be made. Two lightweight-construction student took this upon them self as a bachelors project. 

The new door where constructed in a carbon fibre sandwich style and was open in a scissor manner. The door became sturdier but the hinges where less than ideal and the closing mechanism did not work.

%Results
\section{Results}
We beat Chalmers. We are no byskola.
\subsection{Competition}

\subsection{End of project}

\section{Conclusion}
