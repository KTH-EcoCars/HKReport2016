\chapter{Future Work}
The rules of SEM changes slightly every year and Elba has to be updated accordingly. We 
already know that modifications to the clutch have to be done so the ICE can idle while the car is standing still and new conventional door hinges. Additional parts definitely have to be changed as well before Elba can pass both inspections.

If redoing the clutch how to make faster while more reliable and which modes should be able to engage when must be thoroughly reviewed. It seems that an additional stage have to be added to allow BLDC and ICE be connected while not connected to the wheel. 

The are lots of improvements possible related to the optimisation. And we believe in the proposed master thesis.
%\section{Futurework}

%List of stuff to work on:

%1. Separate ground between logic and power in testrig. Not possible because of the way we %measure voltage. Should share logic ground with ESCON driver.

%2. Map ICE parameters.

%3. Improve parameters of Elba in the plant model.

%4. Local optimization? Not State flow.

%5. Improve clutch

%6. Implement particle filter

%7. Fix CAN receive from RCP to Front ECU

%8. Add more interesting data to CAN msg from RCP

%9. Fix instrumentation panel

%10. Make the testrig to one unit i.e. mount the ECU on the testrig and make the wiring %better.

%11. Make the door hinges more robust\\

%This list of improvements cover all of the different projects and teams that work with Elba %the car. 