\chapter{Discussion}
The purpose of this section is to give the authors thoughts and considerations
on how the different design decisions and project as a whole have turned out
with the benefit of hindsight. Not everything have been clear from the begining
and a lot of lessons have been learned along the way. The easiest way to
communicate this is to start with the project as a whole and then move into the
different parts of the project one by one. 
\section{Elba}

\section{Discussion}

\subsection{Requirements}
Having good requirements gives a common goal for everyone to work towards. At
the start of the project, not much thought was given to setting requirements
first and then work towards fulfilling them. They where instead treated as
implicit to the problems to be solved and an estimation of what was acceptable
end goals where made up as we went along in the project. The lack of real
requirements led to difficulties when the projects from the other subgroups
where incorporated into the final design of the car. For example, the clutch
improvements that were made by the machine design team did not meat the
expectations of the mechatronics team i.e. our implicit requirements were not
the same. Had a predefined set of requirements been set up beforehand, the
consolidation and evalution of the design would have been easiier.

To make result evalutation easier, and to get a clearer definition of done,
requirements where set before work on the testrig started. There were some
difficulties with this since we were customer and the supplier, i.e. user and
developer. To get a clear distinction between the roles, it was decided that as
many user requirements as possible would be set with the SEM rules as a base.
Looking back, this was a good decision. We have been able to source many clear
high-level requirements from the rules, for example the requirement that the
testrig should be able to simulate the full London track. We could then sit down
as a design team and reduce the abstract user requirements to technical system
requirements. This separation of roles has helped us set fair requirements that
set demands on the system design but still allows us to create feasable
definitions of done. 

\subsection{Requirements Engineering system}
When evaluating which software to use to store the requirements, we made
special requirements for how this software should work. It was important to have
traceability, both between requirements and to who has made changes, and that
the system was easy to use. Emphasis was on an easy learning curve. Everyone in
the team has used IBM Rational Doors for requirements engineering in a previous
course, but we all found it cumbersome to use and too packed with features for
what we needed. Since only a core set of features was needed, an alternative
path was taken. Having only the essential features and being able to add new
ones as they are needed has been valuable for the team. Even though not everyone
in the team has experiance with HTML programming, the use of Markdown along with
a standard template for requirements still allow everyone to write new and edit
requirements. 

\subsection{Testrig control}
The desired output of the testrig is the combined external forces that affect
the car at any given time on a track. By construction, this means that the
output is the torque on the rollers is the desired output. Because of time
and hardware constraints, a solution where the current fed to the motor is
instead controlled. In theory, since the motor torque constant of a DC motor is
constant, the torque can be calculated by measuring the current fed to the
motor if lossses in the system are known. In practice, it is difficult to
predict losses accurately and the convoluted way of calculating the torque adds
another layer of possible errors. It also puts rather high demands on how well
the system is modeled; the torque estimation can never be better than the
accuracy of the model. An optimal solution would of course be to fit a torque
sensor to the testrig and run the system on a closed loop with current as the
control variable and torque as the output. However, under the
circumstances, the group considers the open loop current control approach to be
the best choice. Firstly, even with a lower output accuracy, it still gives a
good proof of concept of the testrig and shows that a rolling highway simulation
on Elba can be done succesfully. Secondly, a good platform for future
improvement can be built and expanded upon in future projects. It was never
feasable that a full-featured system could be designed, constructed and
implemented in such a short time scale, so it was important to build a stable
expandable platform and we consider that goal accomplished.

\subsection{Testrig identification}
As discussed in the previous section, the current control approach is not
optimal and should not be a final design for the testrig. Therefore, it has been
important to set accuracy requirements that are "good enough". A model with a
good accuracy is needed for the operation of the current setup. This allows the
concept to be proven and also lets next years team use the testrig as a tool for
getting preliminary data at the start of the project. However, spending too much
time on refining the model would be counterproductive since future versions of
the system will not be as dependent on a high accuracy model. Therefore, the
iterative approach used has been a good way of finding the right balance between
accuracy and time spent. The chosen model has obvious flaws.  Because of the
hardware setup, reliable data for the sine wave input could not be collected and
had to be roughly estimated from the settings used in the DC motor driver. This
means that the sine wave input could not be used for more than a way of
validating model behaviour instead of taking exact measurements.  It is
difficult to know weather the steady state error for the sine wave input is
because of the model or the estimation of the voltage.  It was good however to
see that the model performed well in the area where the testrig started moving
from a stand still. 

One large caveat of the parameter estimation is that it did not take measurement
noise into consideration. A big improvement on the reliability of the data could
be achived by using the statistical methods in \cite{modeling1994}. Usually,
measurement noise is modelled as white noise through a suitable filter. The
iterative process used in the parameter estimation could be continued to include
measurement noise. This is something that could be incorporated into further
iterations of the model identification process used, along with taking the
inductance of the motor into account. Having a flexible process with clear gates
between each model iteration has been very beneficial for the group and has
allowed us to make judgements on how much time to put on improving the models
accuracy.

Little consideration has been taken to quantization errors. The
encoder has a high resolution, 1024 pulses per revolution, which is assumed to
be high enough so that it has a small impact on the fidelity of the results.
Also, the sampling rate was set to 10 Hz. This was the highest that could be
used by Simulinks external mode to get reliable communication with the
microcontroller. It does follow the rule of thumb that the sampling rate should
be 10-30 times faster than the fastest dynamics in the system and the sampled
input and output was exact enouch so that most dynamics where rather smooth.
Both the effect of the sampling rate and the quantization of the measurements
should be investigated mathematically to determine their effect on the data.

It should be noted that having model identification as part of the project has
been very rewarding for us as students. A mechatronic engineer works in the
intersection between programming, electronics and control theory and all of
these have been used in the identification process. Control theory tells us how
model the system dynamics, we implement the measurements on an embedded system
and setup the electrical system to get the data. We have also had to use
knowledge from mechatronics subjects such as Dynamics and Motion control,
Mechatronics basic course and Embedded systems. It is this synthesis of subjects
that makes mechatronics what it is and it has been a great experience. 
\section{Conclusion}
