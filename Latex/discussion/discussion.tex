\chapter{Discussion}
The purpose of this section is to give the authors thoughts and considerations
on how the different design decisions and project as a whole have turned out
with the benefit of hindsight. Not everything have been clear from the beginning
and a lot of lessons have been learned along the way. The easiest way to
communicate this is to start with the project as a whole and then move into the
different parts of the project one by one. 

\section{Elba}
\subsection{Clutch decisions}
Figure~\ref{fig:elba_capvoltage_regen} shows how the capacitor voltage behaves in the different drive modes.
The time that the clutch moves when engaging and disengaging is around 7 seconds which is very long. This was decided because then an endpoint is always reached when disengaging, and it is possible to know exactly how long is necessary 
for a good connection when engaging the clutch. It is clear in this figure that this process consumes a substantial 
amount of energy, and should preferably be kept as short as possible. To improve the performance, one way could be to implement a current sensor and then know that when the current rises to a certain value, then clutch plate is firmly stuck to the drive shaft.

\subsection{ICE and controlling it}
Getting the new engine to work in the the car took the most time compared to what we estimated. Starting on ethanol has proven to be a challenge and we later learned that we had to choke a lot. But getting rid of the bevel gear was a major improvement and although there are vibrations when running nothing mechanical broke due to the uneven torque applied from the engine. 

After the competition we added the ability to control on the lambda value which also improved the overall operation of the engine, and made it easier to get running. Still there is a risk of making the spark plugs wet when choking. 

The power output of the new ICE is larger than previous years (even though the cylinder
 volume is smaller) and this goes against what the previous team recommended us to do. 
Our motivation is that this is not as big of a problem if the ICE is used in the hill of the track. There the BLDC can both apply positive or negative torque in the hill which 
has larger demand on torque from the car than a flat track. 

\section{Optimization}
The optimization was one of the things we really wanted to focus on when the
project started. The group really thought that the optimization problem was of
real importance due to the new track in London. Even so to the extent that the
state of the art report that the group wrote during the spring semester was
dedicated to the optimization problem. Once the report was finished there was a
lot to work on Elba in order to get the car ready for London. This in
combination with the fact that the thesis project was not ready with the lookup
table until a couple of days before the race made the work with the optimization
stagnate. There was simply no time to implement the optimization before it was
time to leave for London. Once we got back from the summer brake the main focus
was on correcting the errors that were found during the race and to get Elba
fully functional. There was also an addition to the workload when the project of
building the Testrig started. There was however work done with getting the
positioning system to work, when the RCP2 arrived the was a lot of work put into
getting the GPS signal to be sent over CAN\@. Once this was accomplished the
implementation structure was also developed and the structure of the system is
completed with all the signals and blocks in the right place.  The blocks
themselves is only working to some degree and retrieving the GPS signal from the
CAN and transferring this to Simulink is not yet completed. The particle filter
will also need to be developed were both the encoder and GPS can contribute to
getting the correct distance traveled on the track. We still believe that the
optimization is of great importance and that a working implementation will
improve the performance in the race. This is also what the simulations is
showing when comparing the results from Section~\ref{sec:opt_sim_static} to the
results in Section~\ref{sec:opt_sim_static}.
\section{Testrig}
\subsection{Requirements}
Having good requirements gives a common goal for everyone to work towards. At
the start of the project, not much thought was given to setting requirements
first and then work towards fulfilling them. They where instead treated as
implicit to the problems to be solved and an estimation of what was acceptable
end goals where made up as we went along in the project. The lack of real
requirements led to difficulties when the projects from the other subgroups
where incorporated into the final design of the car. For example, the clutch
improvements that were made by the machine design team did not meet the
expectations of the mechatronics team i.e.\ our implicit requirements were not
the same. Had a predefined set of requirements been set up beforehand, the
consolidation and evaluation of the design would have been easier.

To make result evaluation easier, and to get a clearer definition of done,
requirements where set before work on the testrig started. There were some
difficulties with this since we were customer and the supplier, i.e.\ user and
developer. To get a clear distinction between the roles, it was decided that as
many user requirements as possible would be set with the SEM rules as a base.
Looking back, this was a good decision. We have been able to source many clear
high-level requirements from the rules, for example the requirement that the
testrig should be able to simulate the full London track. We could then sit down
as a design team and reduce the abstract user requirements to technical system
requirements. This separation of roles has helped us set fair requirements that
set demands on the system design but still allows us to create feasible
definitions of done. 

\subsection{Requirements Engineering system}
When evaluating which software to use to store the requirements, we made
special requirements for how this software should work. It was important to have
traceability, both between requirements and to who has made changes, and that
the system was easy to use. Emphasis was on an easy learning curve. Everyone in
the team has used IBM Rational Doors for requirements engineering in a previous
course, but we all found it cumbersome to use and too packed with features for
what we needed. Since only a core set of features was needed, an alternative
path was taken. Having only the essential features and being able to add new
ones as they are needed has been valuable for the team. Even though not everyone
in the team has experience with HTML programming, the use of Markdown along with
a standard template for requirements still allow everyone to write new and edit
requirements. 

\subsection{Clutch decisions}
Figure~\ref{fig:elba_capvoltage_regen} shows how the capacitor voltage behaves in the different drive modes.
The time that the clutch moves when engaging and disengaging is around 7 seconds which is very long. This was decided because then an endpoint is always reached when disengaging, and it is possible to know exactly how long is necessary 
for a good connection when engaging the clutch. It is clear in this figure that this process consumes a substantial 
amount of energy, and should preferably be kept as short as possible. To improve the performance, one way could be to implement a current sensor and then know that when the current rises to a certain value, then clutch plate is firmly stuck to the drive shaft.

\subsection{Testrig concept design}\label{sec:discussion_testrig_concept}
As described in~\ref{testrig_design_concept} it was chosen to use rollers for the testrig instead of attaching a motor directly to the driving wheel. This design had a number of benefits. Firstly it is easy to shift between testing the car on the testrig and on a real track. This is an important feature, especially during the competition. Secondly it can be used on other cars, for example Sleipner. It is however more difficult to control since it is not guaranteed that there is no slip between the wheel of the car and the roller. This property can be modeled better in the future and overall the team feels that the current concept design for the testrig has the highest potential in the long run.

\subsection{Testrig control}\label{sec:discussion_testrig_control}
As described in~\ref{testrig_design_control} the variable monitored to control the torque is current applied to the motor. This choice was made because of time
and hardware constraints. In theory, since the motor torque constant of a DC motor is
constant, the torque can be calculated by measuring the current fed to the
motor if losses in the system are known. In practice, it is difficult to
predict losses accurately and the convoluted way of calculating the torque adds
another layer of possible errors. It also puts rather high demands on how well
the system is modelled; the torque estimation can never be better than the
accuracy of the model. An optimal solution would of course be to fit a torque
sensor to the testrig and run the system on a closed loop with current as the
control variable and torque as the output. However, under the
circumstances, the group considers the open loop current control approach to be
the best choice. Even with a lower output accuracy, it still gives a
good proof of concept of the testrig and shows that a rolling highway simulation
on Elba can be done successfully.
For future improvement of the testrig, torque control should be considered. It is however not deemed to be of the highest priority. %TODO: omformulera?

%Stryka helt?: Secondly, a good platform for future improvement can be built and expanded upon in future projects. It was never feasible that a full-featured system could be designed, constructed and implemented in such a short time scale, so it was important to build a stable expandable platform and we consider that goal accomplished.

\subsection{Testrig identification}
As discussed in the previous section, the current control approach is not
optimal and it should be considered to use another final design for the control of the testrig. Therefore, it has been
important to set accuracy requirements that are "good enough". A model with a
good accuracy is needed for the operation of the current setup. This allows the
concept to be proven and also lets next years team use the testrig as a tool for
getting preliminary data at the start of the project. However, spending too much
time on refining the model would be counterproductive since future versions of
the system will not be as dependent on a high accuracy model. Therefore, the
iterative approach used has been a good way of finding the right balance between
accuracy and time spent. The chosen model has obvious flaws.  Because of the
hardware setup, reliable data for the sine wave input could not be collected and
had to be roughly estimated from the settings used in the DC motor driver. This
means that the sine wave input could not be used for more than a way of
validating model behaviour instead of taking exact measurements.  It is
difficult to know whether the steady state error for the sine wave input is
because of the model or the estimation of the voltage.  It was good however to
see that the model performed well in the area where the testrig started moving
from a stand still. 

One large caveat of the parameter estimation is that it did not take measurement
noise into consideration. A big improvement on the reliability of the data could
be achieved by using the statistical methods in~\cite{modeling1994}. Usually,
measurement noise is modelled as white noise through a suitable filter. The
iterative process used in the parameter estimation could be continued to include
measurement noise. This is something that could be incorporated into further
iterations of the model identification process used, along with taking the
inductance of the motor into account. Having a flexible process with clear gates
between each model iteration has been very beneficial for the group and has
allowed us to make judgements on how much time to put on improving the models
accuracy.

Little consideration has been taken to quantization errors. The
encoder has a high resolution, 1024 pulses per revolution, which is assumed to
be high enough so that it has a small impact on the fidelity of the results.
Also, the sampling rate was set to 10 Hz. This was the highest that could be
used by Simulinks external mode to get reliable communication with the
microcontroller. It does follow the rule of thumb that the sampling rate should
be 10--30 times faster than the fastest dynamics in the system and the sampled
input and output was exact enough so that most dynamics where rather smooth.
Both the effect of the sampling rate and the quantization of the measurements
should be investigated mathematically to determine their effect on the data.

It should be noted that having model identification as part of the project has
been very rewarding for us as students. A mechatronic engineer works in the
intersection between programming, electronics and control theory and all of
these have been used in the identification process. Control theory tells us how
model the system dynamics, we implement the measurements on an embedded system
and setup the electrical system to get the data. We have also had to use
knowledge from mechatronics subjects such as Dynamics and Motion control,
Mechatronics basic course and Embedded systems. It is this synthesis of subjects
that makes mechatronics what it is and it has been a great experience. 

\subsection{Energy dissipation}
The energy dissipation is working reasonably well considering it was made during the final part of the project. The biggest flaw at the moment is that the microcontroller crashes due to disturbances, and this can eventually be a safety hazard if the supercapacitor is overcharged to dangerous levels. More work to get rid of or handle the disturbances better is needed. Extra safety features could also be included, which prevents the supercapacitor from getting overcharged. 

The power supply will rarely be used, since the accumulated energy in the  supercapacitor will in most casesbe enough to simulate downhills. Energy in the supercapacitor is also used for powering the the ESCON driver as well as the microcontroller, which is good since the generated energy is used as efficiently as possible. 
